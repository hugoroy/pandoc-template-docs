\usepackage[pagestyles,psfloats]{titlesec}
\usepackage{textcase}
% optional
% \usepackage{varioref} necessary in your preamble for \vparnum to work
% \usepackage{wallpaper}
%% % If you have a cover/background
%% \ThisCenterWallPaper{1}{file.pdf}

%%%%%%%%%%%%%%%%% EXTRA FONT OPTIONS %%%%%%%%%%%%%%%%%
%% % Access styles
%% \ifnum 0\ifxetex 1\fi\ifluatex 1\fi=0 % if pdftex
%%     \else % if luatex or xetex
%%         \newfontfamily\sbweight{Fontname Semi Bold}
%%         \newfontfamily\blackweight{Fontname Black}
%%         % If font has no small caps
%%         \renewcommand{\textsc}{\MakeTextUppercase}
%% \fi

%%%%%%%%%%%%%%%%% BODY %%%%%%%%%%%%%%%%%

%\parskip=8.25pt
%\setlength{\parskip}{6pt plus 2pt minus 1pt}}

\widowpenalty10000
\clubpenalty10000

%%%%%%%%%%%% ENUMERATE LISTS %%%%%%%%%%%%

\usepackage[shortlabels]{enumitem}
\setlist{labelsep=7pt}

% Use tabular numbers for the enumerated list marker
\ifnum 0\ifxetex 1\fi\ifluatex 1\fi=0 % if pdftex
    \else % if luatex or xetex
        \setlist[enumerate]{font=\addfontfeatures{Numbers=Monospaced}}
\fi

%%%%%%%%%%%% TABLES %%%%%%%%%%%%

% Use tabular numbers in tables
\ifnum 0\ifxetex 1\fi\ifluatex 1\fi=0 % if pdftex
    \else % if luatex or xetex
        \let\oldtable\table
        \let\endoldtable\endtable
        \renewenvironment{table}{
            \begin{oldtable}
              \addfontfeatures{Numbers=Monospaced}
            }
            {\end{oldtable}}
\fi

%%%%%%%%%%%%%%%%% TITLE %%%%%%%%%%%%%%%%% 

% Use the \title of the document in places other than \maketitle (e.g. headers).
% Specify an alternative title to be used with \renewcommand\doctitle{}
\ifx\@title\undefined
    % The document title may be used beyond LaTeX' \maketitle,
    % e.g. for headers, don't forget to \renewcommand\doctitle
    % if no title is set
    \newcommand{\doctitle}{}
    \else
        \let\oldtitle\title
        \newcommand{\doctitle}{}
        \renewcommand{\title}[1]{%
        % Display title in small caps
        \oldtitle{\scshape\MakeTextLowercase{#1}}%
        % Use the title variable 
        \renewcommand{\doctitle}{#1}}
\fi


%%%%%%%%%% TABLE OF CONTENTS %%%%%%%%%%%%

% Place table of content at the bottom of the page and clear
% for a new opening page after
\let\oldtableofcontents\tableofcontents
\renewcommand\tableofcontents{\vfill \oldtableofcontents \cleardoublepage}

%%%%%%%%%%%%%%%%% HEADINGS %%%%%%%%%%%%%%%%%%

%% \renewcommand{\thepart}{\Roman{part}}
%\titleclass{\part}{page}
%\assignpagestyle{\part}{plain}

%%\renewcommand{\thechapter}{\Roman{chapter}}
%% \titleformat{\chapter}[display]{\centering\Large}{}{0pt}{%
%% \thechapter. \textsc}
%% \titleformat{name=\chapter,numberless}[display]{\centering\Large}{}{0pt}{%
%% \textsc}
%%\titlespacing{\chapter}{3pc}{*3}{*2}[3pc]
%%\titleclass{\chapter}{straight}
%% \titleclass{\chapter}{top}
%%\assignpagestyle{\chapter}{plain}

% For Faits/Discussion type chapters in court filings
\titleformat{\chapter}[display]{\centering\bfseries}{}{0pt}{}
\titleformat{name=\chapter,numberless}[display]{\centering\bfseries}{}{0pt}{%
}
\titlespacing{\chapter}{\parindent}{9pt}{9pt}[\parindent]
%%\assignpagestyle{\chapter}{plain}

\titleformat{\section}[display]{\large\bfseries}{}{0pt}{\strut\llap{\thesection.~}}
\titleformat{name=\section,numberless}[display]{\large\bfseries}{}{0pt}{}
\titlespacing{\section}{\parindent}{9pt}{9pt}[\parindent]

\renewcommand{\thesection}{\arabic{section}}

\titleformat{\subsection}[display]{\bfseries}{}{0pt}{\strut\llap{\small\thesubsection.~}}
\titleformat{name=\subsection,numberless}[display]{\bfseries}{}{0pt}{}
\titlespacing{\subsection}{\parindent}{2pt}{6pt}[10pt]

\titleformat{\subsubsection}[display]{\bfseries\small}{}{0pt}{\thesubsubsection.~}
\titleformat{name=\subsubsection,numberless}[display]{\bfseries\small}{}{0pt}{}
\titlespacing{\subsubsection}{\parindent}{0pt}{0pt}[10pt]

\titleformat{\paragraph}[display]{\bfseries\small}{\quad}{0pt}{\theparagraph.~}[]
\titleformat{name=\paragraph,numberless}[display]{\bfseries\small}{}{0pt}{}[]
\titlespacing{\paragraph}{0pt}{10pt}{\parindent}[\parindent]

\titleformat{\subparagraph}[runin]{\bfseries\small}{}{0pt}{\thesubparagraph.~}[.\enspace]
\titleformat{name=\subparagraph,numberless}[runin]{\bfseries\small}{}{0pt}{}[.\enspace]
\titlespacing{\subparagraph}{\parindent}{10pt}{\parindent}[\parindent]

\titleformat{\subsubparagraph}[runin]{\slshape\small}{}{0pt}{\thesubsubparagraph.~}[.\enspace]
\titleformat{name=\subsubparagraph,numberless}[runin]{\slshape\small}{}{0pt}{}[.\enspace]
\titlespacing{\subsubparagraph}{\parindent}{10pt}{\parindent}[\parindent]


%%%%%%%%%%%%%%%%% HEADS & FOOTS %%%%%%%%%%%%%%%%%

%% % Chose your page style
%% \pagestyle{withdoctitle}

\newpagestyle{main}[\footnotesize\scshape]{
    \sethead[\normalsize\llap{\thepage}~][][]
            {}{}{\normalsize\rlap{\thepage}~}
    }

\newpagestyle{withdoctitle}[\small\scshape]{
\sethead[\normalsize\llap{\thepage
        }~][
        ][]  % even
        {\MakeTextLowercase{\doctitle}}{
        }{\rlap{\thepage
        }~}} % odd
     
\newpagestyle{chapter}[\small\scshape]{
\sethead[\normalsize\llap{\thepage
        }~][
        ][]  % even
        {\MakeTextLowercase{}}{
        }{\rlap{\thepage
        }~}} % odd

\newpagestyle{back}[]{
\sethead[][][]{}{}{}
\setfoot[][\thepage /\thepage][]{}{\thepage /\thepage}{}%{\rlap{\thepage}}}% odd
}
     
%% % If a logo is used
%% \usepackage{graphicx}
%% \newpagestyle{withlogo}[\footnotesize\scshape]{
%% \sethead[][][]  % even
%% {\llap{\includegraphics[height=0.25cm]{logo.pdf}\hspace{0.07cm}}\raisebox{-0.35cm}{\includegraphics[height=0.6cm]{logotype.pdf}}}{\doctitle}{} % odd
%% \setfoot[\small\llap{\thepage}~][][]{}{}{\small\rlap{\thepage}~}
%% }

%%%%%%%%%%%%%%%%% STUFF %%%%%%%%%%%%%%%%%

% Hang quotes
\def\Hangquote{\par\ifhmode\ERROR\fi\leavevmode\llap}                         

% Number paragraphs
\newcommand{\para}{}
\newcounter{paranumero}
\newif\ifcounting\countingtrue
\renewcommand{\para}[1]{\noindent\ifcounting\refstepcounter{paranumero}\fi{\small\leavevmode\llap{\textsf{\scriptsize
\ifnum 0\ifxetex 1\fi\ifluatex 1\fi=0 % if pdftex
    \else % if luatex or xetex
        \setlist[enumerate]{font=\addfontfeatures{Numbers=Monospaced}}
\fi
        \theparanumero}\quad\quad}}\label{#1}}


\newenvironment{noparanum}{
  \renewcommand\para{}
}


% Ref to numbered paragraphs
\newcommand{\parnum}[1]{¶\ref{#1}}
\newcommand{\vparnum}[1]{¶\vref{#1}}

% Block quotes
\let\oldquote\quote
\let\endoldquote\endquote
\renewenvironment{quote}{
  \vskip 12pt \begin{oldquote}
        \ifnum 0\ifxetex 1\fi\ifluatex 1\fi=0 % if pdftex
        \renewcommand{\para}[1]{}\renewcommand\sfdefault{lmssq}\sffamily\scriptsize
          \else % if luatex or xetex
          % define a quotefont in your preamble with \newfontfamily\quotefont[Options]{Font}
          \renewcommand{\para}[1]{}
          \ifx\quotefont\undefined \else \quotefont \fi
        \fi
        }
  {\end{oldquote}\vskip 14pt}


%%%%%%%%%%%%%%% FOOTNOTES %%%%%%%%%%%%%%%%%

% Don't show the footnoterule use whitespace
\renewcommand\footnoterule{\vskip 16pt}
\let\oldfootnote\footnote
\renewcommand\footnote[1]{\oldfootnote{\renewcommand\para{}#1}}


